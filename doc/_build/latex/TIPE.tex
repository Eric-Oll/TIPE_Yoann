%% Generated by Sphinx.
\def\sphinxdocclass{report}
\documentclass[a4paper,10pt,french]{sphinxmanual}
\ifdefined\pdfpxdimen
   \let\sphinxpxdimen\pdfpxdimen\else\newdimen\sphinxpxdimen
\fi \sphinxpxdimen=.75bp\relax

\usepackage[utf8]{inputenc}
\ifdefined\DeclareUnicodeCharacter
 \ifdefined\DeclareUnicodeCharacterAsOptional
  \DeclareUnicodeCharacter{"00A0}{\nobreakspace}
  \DeclareUnicodeCharacter{"2500}{\sphinxunichar{2500}}
  \DeclareUnicodeCharacter{"2502}{\sphinxunichar{2502}}
  \DeclareUnicodeCharacter{"2514}{\sphinxunichar{2514}}
  \DeclareUnicodeCharacter{"251C}{\sphinxunichar{251C}}
  \DeclareUnicodeCharacter{"2572}{\textbackslash}
 \else
  \DeclareUnicodeCharacter{00A0}{\nobreakspace}
  \DeclareUnicodeCharacter{2500}{\sphinxunichar{2500}}
  \DeclareUnicodeCharacter{2502}{\sphinxunichar{2502}}
  \DeclareUnicodeCharacter{2514}{\sphinxunichar{2514}}
  \DeclareUnicodeCharacter{251C}{\sphinxunichar{251C}}
  \DeclareUnicodeCharacter{2572}{\textbackslash}
 \fi
\fi
\usepackage{cmap}
\usepackage[T1]{fontenc}
\usepackage{amsmath,amssymb,amstext}
\usepackage{babel}
\usepackage{times}
\usepackage[Sonny]{fncychap}
\usepackage[dontkeepoldnames]{sphinx}

\usepackage{geometry}

% Include hyperref last.
\usepackage{hyperref}
% Fix anchor placement for figures with captions.
\usepackage{hypcap}% it must be loaded after hyperref.
% Set up styles of URL: it should be placed after hyperref.
\urlstyle{same}
\addto\captionsfrench{\renewcommand{\contentsname}{Contents:}}

\addto\captionsfrench{\renewcommand{\figurename}{Fig.}}
\addto\captionsfrench{\renewcommand{\tablename}{Tableau}}
\addto\captionsfrench{\renewcommand{\literalblockname}{Code source}}

\addto\captionsfrench{\renewcommand{\literalblockcontinuedname}{continued from previous page}}
\addto\captionsfrench{\renewcommand{\literalblockcontinuesname}{continues on next page}}

\addto\extrasfrench{\def\pageautorefname{page}}

\setcounter{tocdepth}{1}



\title{TIPE Documentation}
\date{mars 14, 2021}
\release{1}
\author{Eric}
\newcommand{\sphinxlogo}{\vbox{}}
\renewcommand{\releasename}{Version}
\makeindex

\begin{document}

\maketitle
\sphinxtableofcontents
\phantomsection\label{\detokenize{index::doc}}



\chapter{Package \sphinxtitleref{road\_objects}}
\label{\detokenize{road_objects/package:documentation-sur-le-projet-tipe}}\label{\detokenize{road_objects/package::doc}}\label{\detokenize{road_objects/package:package-road-objects}}

\section{Module graphical\_item}
\label{\detokenize{road_objects/graphical_item:module-graphical-item}}\label{\detokenize{road_objects/graphical_item::doc}}\label{\detokenize{road_objects/graphical_item:module-road_objects.graphical_item}}\index{road\_objects.graphical\_item (module)}

\subsection{Module \sphinxtitleref{graphical\_item.py}}
\label{\detokenize{road_objects/graphical_item:module-graphical-item-py}}
\sphinxstyleemphasis{Created on} 21/02/2021 \sphinxstyleemphasis{by} Eric Ollivier

\sphinxstyleemphasis{Versioning :}
- 0.1 : Initial version
\index{GraphicalItem (classe dans road\_objects.graphical\_item)}

\begin{fulllineitems}
\phantomsection\label{\detokenize{road_objects/graphical_item:road_objects.graphical_item.GraphicalItem}}\pysiglinewithargsret{\sphinxbfcode{class }\sphinxcode{road\_objects.graphical\_item.}\sphinxbfcode{GraphicalItem}}{\emph{axe: matplotlib.axes.\_axes.Axes}}{}
Représentation graphique des objets de la route
\begin{quote}\begin{description}
\item[{Paramètres}] \leavevmode
\sphinxstyleliteralstrong{axe} \textendash{} objet de type matplotlib.Axes (contexe graphique)

\end{description}\end{quote}
\index{\_\_class\_\_ (attribut road\_objects.graphical\_item.GraphicalItem)}

\begin{fulllineitems}
\phantomsection\label{\detokenize{road_objects/graphical_item:road_objects.graphical_item.GraphicalItem.__class__}}\pysigline{\sphinxbfcode{\_\_class\_\_}}
alias de \sphinxcode{type}

\end{fulllineitems}

\index{\_\_delattr\_\_ (attribut road\_objects.graphical\_item.GraphicalItem)}

\begin{fulllineitems}
\phantomsection\label{\detokenize{road_objects/graphical_item:road_objects.graphical_item.GraphicalItem.__delattr__}}\pysigline{\sphinxbfcode{\_\_delattr\_\_}}
Implement delattr(self, name).

\end{fulllineitems}

\index{\_\_dict\_\_ (attribut road\_objects.graphical\_item.GraphicalItem)}

\begin{fulllineitems}
\phantomsection\label{\detokenize{road_objects/graphical_item:road_objects.graphical_item.GraphicalItem.__dict__}}\pysigline{\sphinxbfcode{\_\_dict\_\_}\sphinxbfcode{ = mappingproxy(\{'\_\_module\_\_': 'road\_objects.graphical\_item', '\_\_doc\_\_': '\textbackslash{}n    Représentation graphique des objets de la route\textbackslash{}n    ', '\_\_init\_\_': \textless{}function GraphicalItem.\_\_init\_\_\textgreater{}, 'ax': \textless{}property object\textgreater{}, '\_\_getitem\_\_': \textless{}function GraphicalItem.\_\_getitem\_\_\textgreater{}, 'add\_component': \textless{}function GraphicalItem.add\_component\textgreater{}, 'add\_plot': \textless{}function GraphicalItem.add\_plot\textgreater{}, 'get\_components': \textless{}function GraphicalItem.get\_components\textgreater{}, '\_\_dict\_\_': \textless{}attribute '\_\_dict\_\_' of 'GraphicalItem' objects\textgreater{}, '\_\_weakref\_\_': \textless{}attribute '\_\_weakref\_\_' of 'GraphicalItem' objects\textgreater{}\})}}
\end{fulllineitems}

\index{\_\_dir\_\_() (méthode road\_objects.graphical\_item.GraphicalItem)}

\begin{fulllineitems}
\phantomsection\label{\detokenize{road_objects/graphical_item:road_objects.graphical_item.GraphicalItem.__dir__}}\pysiglinewithargsret{\sphinxbfcode{\_\_dir\_\_}}{}{{ $\rightarrow$ list}}
default dir() implementation

\end{fulllineitems}

\index{\_\_eq\_\_ (attribut road\_objects.graphical\_item.GraphicalItem)}

\begin{fulllineitems}
\phantomsection\label{\detokenize{road_objects/graphical_item:road_objects.graphical_item.GraphicalItem.__eq__}}\pysigline{\sphinxbfcode{\_\_eq\_\_}}
Return self==value.

\end{fulllineitems}

\index{\_\_format\_\_() (méthode road\_objects.graphical\_item.GraphicalItem)}

\begin{fulllineitems}
\phantomsection\label{\detokenize{road_objects/graphical_item:road_objects.graphical_item.GraphicalItem.__format__}}\pysiglinewithargsret{\sphinxbfcode{\_\_format\_\_}}{}{}
default object formatter

\end{fulllineitems}

\index{\_\_ge\_\_ (attribut road\_objects.graphical\_item.GraphicalItem)}

\begin{fulllineitems}
\phantomsection\label{\detokenize{road_objects/graphical_item:road_objects.graphical_item.GraphicalItem.__ge__}}\pysigline{\sphinxbfcode{\_\_ge\_\_}}
Return self\textgreater{}=value.

\end{fulllineitems}

\index{\_\_getattribute\_\_ (attribut road\_objects.graphical\_item.GraphicalItem)}

\begin{fulllineitems}
\phantomsection\label{\detokenize{road_objects/graphical_item:road_objects.graphical_item.GraphicalItem.__getattribute__}}\pysigline{\sphinxbfcode{\_\_getattribute\_\_}}
Return getattr(self, name).

\end{fulllineitems}

\index{\_\_getitem\_\_() (méthode road\_objects.graphical\_item.GraphicalItem)}

\begin{fulllineitems}
\phantomsection\label{\detokenize{road_objects/graphical_item:road_objects.graphical_item.GraphicalItem.__getitem__}}\pysiglinewithargsret{\sphinxbfcode{\_\_getitem\_\_}}{\emph{name}}{}
\end{fulllineitems}

\index{\_\_gt\_\_ (attribut road\_objects.graphical\_item.GraphicalItem)}

\begin{fulllineitems}
\phantomsection\label{\detokenize{road_objects/graphical_item:road_objects.graphical_item.GraphicalItem.__gt__}}\pysigline{\sphinxbfcode{\_\_gt\_\_}}
Return self\textgreater{}value.

\end{fulllineitems}

\index{\_\_hash\_\_ (attribut road\_objects.graphical\_item.GraphicalItem)}

\begin{fulllineitems}
\phantomsection\label{\detokenize{road_objects/graphical_item:road_objects.graphical_item.GraphicalItem.__hash__}}\pysigline{\sphinxbfcode{\_\_hash\_\_}}
Return hash(self).

\end{fulllineitems}

\index{\_\_init\_\_() (méthode road\_objects.graphical\_item.GraphicalItem)}

\begin{fulllineitems}
\phantomsection\label{\detokenize{road_objects/graphical_item:road_objects.graphical_item.GraphicalItem.__init__}}\pysiglinewithargsret{\sphinxbfcode{\_\_init\_\_}}{\emph{axe: matplotlib.axes.\_axes.Axes}}{}~\begin{quote}\begin{description}
\item[{Paramètres}] \leavevmode
\sphinxstyleliteralstrong{axe} \textendash{} objet de type matplotlib.Axes (contexe graphique)

\end{description}\end{quote}

\end{fulllineitems}

\index{\_\_init\_subclass\_\_() (méthode road\_objects.graphical\_item.GraphicalItem)}

\begin{fulllineitems}
\phantomsection\label{\detokenize{road_objects/graphical_item:road_objects.graphical_item.GraphicalItem.__init_subclass__}}\pysiglinewithargsret{\sphinxbfcode{\_\_init\_subclass\_\_}}{}{}
This method is called when a class is subclassed.

The default implementation does nothing. It may be
overridden to extend subclasses.

\end{fulllineitems}

\index{\_\_le\_\_ (attribut road\_objects.graphical\_item.GraphicalItem)}

\begin{fulllineitems}
\phantomsection\label{\detokenize{road_objects/graphical_item:road_objects.graphical_item.GraphicalItem.__le__}}\pysigline{\sphinxbfcode{\_\_le\_\_}}
Return self\textless{}=value.

\end{fulllineitems}

\index{\_\_lt\_\_ (attribut road\_objects.graphical\_item.GraphicalItem)}

\begin{fulllineitems}
\phantomsection\label{\detokenize{road_objects/graphical_item:road_objects.graphical_item.GraphicalItem.__lt__}}\pysigline{\sphinxbfcode{\_\_lt\_\_}}
Return self\textless{}value.

\end{fulllineitems}

\index{\_\_module\_\_ (attribut road\_objects.graphical\_item.GraphicalItem)}

\begin{fulllineitems}
\phantomsection\label{\detokenize{road_objects/graphical_item:road_objects.graphical_item.GraphicalItem.__module__}}\pysigline{\sphinxbfcode{\_\_module\_\_}\sphinxbfcode{ = 'road\_objects.graphical\_item'}}
\end{fulllineitems}

\index{\_\_ne\_\_ (attribut road\_objects.graphical\_item.GraphicalItem)}

\begin{fulllineitems}
\phantomsection\label{\detokenize{road_objects/graphical_item:road_objects.graphical_item.GraphicalItem.__ne__}}\pysigline{\sphinxbfcode{\_\_ne\_\_}}
Return self!=value.

\end{fulllineitems}

\index{\_\_new\_\_() (méthode road\_objects.graphical\_item.GraphicalItem)}

\begin{fulllineitems}
\phantomsection\label{\detokenize{road_objects/graphical_item:road_objects.graphical_item.GraphicalItem.__new__}}\pysiglinewithargsret{\sphinxbfcode{\_\_new\_\_}}{}{}
Create and return a new object.  See help(type) for accurate signature.

\end{fulllineitems}

\index{\_\_reduce\_\_() (méthode road\_objects.graphical\_item.GraphicalItem)}

\begin{fulllineitems}
\phantomsection\label{\detokenize{road_objects/graphical_item:road_objects.graphical_item.GraphicalItem.__reduce__}}\pysiglinewithargsret{\sphinxbfcode{\_\_reduce\_\_}}{}{}
helper for pickle

\end{fulllineitems}

\index{\_\_reduce\_ex\_\_() (méthode road\_objects.graphical\_item.GraphicalItem)}

\begin{fulllineitems}
\phantomsection\label{\detokenize{road_objects/graphical_item:road_objects.graphical_item.GraphicalItem.__reduce_ex__}}\pysiglinewithargsret{\sphinxbfcode{\_\_reduce\_ex\_\_}}{}{}
helper for pickle

\end{fulllineitems}

\index{\_\_repr\_\_ (attribut road\_objects.graphical\_item.GraphicalItem)}

\begin{fulllineitems}
\phantomsection\label{\detokenize{road_objects/graphical_item:road_objects.graphical_item.GraphicalItem.__repr__}}\pysigline{\sphinxbfcode{\_\_repr\_\_}}
Return repr(self).

\end{fulllineitems}

\index{\_\_setattr\_\_ (attribut road\_objects.graphical\_item.GraphicalItem)}

\begin{fulllineitems}
\phantomsection\label{\detokenize{road_objects/graphical_item:road_objects.graphical_item.GraphicalItem.__setattr__}}\pysigline{\sphinxbfcode{\_\_setattr\_\_}}
Implement setattr(self, name, value).

\end{fulllineitems}

\index{\_\_sizeof\_\_() (méthode road\_objects.graphical\_item.GraphicalItem)}

\begin{fulllineitems}
\phantomsection\label{\detokenize{road_objects/graphical_item:road_objects.graphical_item.GraphicalItem.__sizeof__}}\pysiglinewithargsret{\sphinxbfcode{\_\_sizeof\_\_}}{}{{ $\rightarrow$ int}}
size of object in memory, in bytes

\end{fulllineitems}

\index{\_\_str\_\_ (attribut road\_objects.graphical\_item.GraphicalItem)}

\begin{fulllineitems}
\phantomsection\label{\detokenize{road_objects/graphical_item:road_objects.graphical_item.GraphicalItem.__str__}}\pysigline{\sphinxbfcode{\_\_str\_\_}}
Return str(self).

\end{fulllineitems}

\index{\_\_subclasshook\_\_() (méthode road\_objects.graphical\_item.GraphicalItem)}

\begin{fulllineitems}
\phantomsection\label{\detokenize{road_objects/graphical_item:road_objects.graphical_item.GraphicalItem.__subclasshook__}}\pysiglinewithargsret{\sphinxbfcode{\_\_subclasshook\_\_}}{}{}
Abstract classes can override this to customize issubclass().

This is invoked early on by abc.ABCMeta.\_\_subclasscheck\_\_().
It should return True, False or NotImplemented.  If it returns
NotImplemented, the normal algorithm is used.  Otherwise, it
overrides the normal algorithm (and the outcome is cached).

\end{fulllineitems}

\index{\_\_weakref\_\_ (attribut road\_objects.graphical\_item.GraphicalItem)}

\begin{fulllineitems}
\phantomsection\label{\detokenize{road_objects/graphical_item:road_objects.graphical_item.GraphicalItem.__weakref__}}\pysigline{\sphinxbfcode{\_\_weakref\_\_}}
list of weak references to the object (if defined)

\end{fulllineitems}

\index{add\_component() (méthode road\_objects.graphical\_item.GraphicalItem)}

\begin{fulllineitems}
\phantomsection\label{\detokenize{road_objects/graphical_item:road_objects.graphical_item.GraphicalItem.add_component}}\pysiglinewithargsret{\sphinxbfcode{add\_component}}{\emph{name: str}, \emph{component}}{}
Ajout un élement graphique de base pour la représentation de l’Item
\begin{quote}\begin{description}
\item[{Paramètres}] \leavevmode\begin{itemize}
\item {} 
\sphinxstyleliteralstrong{name} \textendash{} Nom du composant

\item {} 
\sphinxstyleliteralstrong{component} \textendash{} Composant à ajouter de type matplotlib.Artist

\end{itemize}

\end{description}\end{quote}

\end{fulllineitems}

\index{add\_plot() (méthode road\_objects.graphical\_item.GraphicalItem)}

\begin{fulllineitems}
\phantomsection\label{\detokenize{road_objects/graphical_item:road_objects.graphical_item.GraphicalItem.add_plot}}\pysiglinewithargsret{\sphinxbfcode{add\_plot}}{\emph{*args}, \emph{name: str}, \emph{**kwargs}}{}
Crée un objet grapghique avec la méthode \sphinxtitleref{Axes.plot} et l’ajoute  à la liste des composants graphiques
\begin{quote}\begin{description}
\item[{Paramètres}] \leavevmode\begin{itemize}
\item {} 
\sphinxstyleliteralstrong{name} \textendash{} Nom à donner au nouveau composant

\item {} 
\sphinxstyleliteralstrong{args} \textendash{} Liste des paramètre pour la méthode \sphinxtitleref{Axes.plot}

\item {} 
\sphinxstyleliteralstrong{kwargs} \textendash{} Liste des paramètres nommés pour la méthode \sphinxtitleref{Axes.plot}

\end{itemize}

\end{description}\end{quote}

\end{fulllineitems}

\index{ax (attribut road\_objects.graphical\_item.GraphicalItem)}

\begin{fulllineitems}
\phantomsection\label{\detokenize{road_objects/graphical_item:road_objects.graphical_item.GraphicalItem.ax}}\pysigline{\sphinxbfcode{ax}}
\end{fulllineitems}

\index{get\_components() (méthode road\_objects.graphical\_item.GraphicalItem)}

\begin{fulllineitems}
\phantomsection\label{\detokenize{road_objects/graphical_item:road_objects.graphical_item.GraphicalItem.get_components}}\pysiglinewithargsret{\sphinxbfcode{get\_components}}{}{{ $\rightarrow$ list}}
Fournit la liste de composants graphique
\begin{quote}\begin{description}
\item[{Retourne}] \leavevmode
objet \sphinxtitleref{list} d’objets de type matplotlib.Artist

\end{description}\end{quote}

\end{fulllineitems}


\end{fulllineitems}



\section{Module road\_item}
\label{\detokenize{road_objects/road_item::doc}}\label{\detokenize{road_objects/road_item:module-road_objects.road_item}}\label{\detokenize{road_objects/road_item:module-road-item}}\index{road\_objects.road\_item (module)}

\subsection{Module name : \sphinxtitleref{road\_item.py}}
\label{\detokenize{road_objects/road_item:module-name-road-item-py}}
\sphinxstyleemphasis{Created on} 14/02/2021 \sphinxstyleemphasis{by} Eric Ollivier

\sphinxstyleemphasis{Versionning :}
\begin{itemize}
\item {} 
0.1 : Initial version

\item {} 
0.2 ; Ajout de la notion de franchissable (attribut “passable”)

\end{itemize}
\index{RoadItem (classe dans road\_objects.road\_item)}

\begin{fulllineitems}
\phantomsection\label{\detokenize{road_objects/road_item:road_objects.road_item.RoadItem}}\pysiglinewithargsret{\sphinxbfcode{class }\sphinxcode{road\_objects.road\_item.}\sphinxbfcode{RoadItem}}{\emph{axe}, \emph{path=None}, \emph{roads=None}, \emph{name=None}, \emph{**kwargs}}{}
Classe de base pour les objets de la route :
\begin{itemize}
\item {} 
Véhicule

\item {} 
signalisation (ex : feux tricolores)

\item {} 
obstable (à venir)

\item {} 
…

\end{itemize}

Contribue à calculer le changement de vitesse des véhicules
\begin{quote}\begin{description}
\item[{Paramètres}] \leavevmode\begin{itemize}
\item {} 
\sphinxstyleliteralstrong{axe} \textendash{} Context graphique (objet matplotlib.Axes)

\item {} 
\sphinxstyleliteralstrong{path} \textendash{} Itinéraire de l’objet \sphinxtitleref{RoadItem} (objet de type \sphinxtitleref{roadmaps.Path})

\item {} 
\sphinxstyleliteralstrong{roads} \textendash{} Liste des routes si \sphinxtitleref{path} n’est pas défini(objet \sphinxtitleref{list} d’objet \sphinxtitleref{roadmap.Road})

\item {} 
\sphinxstyleliteralstrong{name} \textendash{} Nom de l’objet \sphinxtitleref{RoadItem}

\item {} 
\sphinxstyleliteralstrong{current\_time} \textendash{} Valeur initial du temps courant

\item {} 
\sphinxstyleliteralstrong{init\_time} \textendash{} Valeur du temps de début de scénario

\item {} 
\sphinxstyleliteralstrong{index} \textendash{} Valeur initial de la position dans l’itinéraire

\item {} 
\sphinxstyleliteralstrong{passable} \textendash{} Valeur initial de propriété de franchissement

\end{itemize}

\end{description}\end{quote}
\index{Add\_item() (méthode de la classe road\_objects.road\_item.RoadItem)}

\begin{fulllineitems}
\phantomsection\label{\detokenize{road_objects/road_item:road_objects.road_item.RoadItem.Add_item}}\pysiglinewithargsret{\sphinxbfcode{classmethod }\sphinxbfcode{Add\_item}}{\emph{item}}{}
Ajoute un objet à l’inventaire des items de la route

\end{fulllineitems}

\index{Get\_Items() (méthode de la classe road\_objects.road\_item.RoadItem)}

\begin{fulllineitems}
\phantomsection\label{\detokenize{road_objects/road_item:road_objects.road_item.RoadItem.Get_Items}}\pysiglinewithargsret{\sphinxbfcode{classmethod }\sphinxbfcode{Get\_Items}}{}{}
Retourne la liste des items

\end{fulllineitems}

\index{\_\_eq\_\_() (méthode road\_objects.road\_item.RoadItem)}

\begin{fulllineitems}
\phantomsection\label{\detokenize{road_objects/road_item:road_objects.road_item.RoadItem.__eq__}}\pysiglinewithargsret{\sphinxbfcode{\_\_eq\_\_}}{\emph{other}}{}
Définit l’égalité pour la comparaison entre les objets

\end{fulllineitems}

\index{\_\_init\_\_() (méthode road\_objects.road\_item.RoadItem)}

\begin{fulllineitems}
\phantomsection\label{\detokenize{road_objects/road_item:road_objects.road_item.RoadItem.__init__}}\pysiglinewithargsret{\sphinxbfcode{\_\_init\_\_}}{\emph{axe}, \emph{path=None}, \emph{roads=None}, \emph{name=None}, \emph{**kwargs}}{}~\begin{quote}\begin{description}
\item[{Paramètres}] \leavevmode\begin{itemize}
\item {} 
\sphinxstyleliteralstrong{axe} \textendash{} Context graphique (objet matplotlib.Axes)

\item {} 
\sphinxstyleliteralstrong{path} \textendash{} Itinéraire de l’objet \sphinxtitleref{RoadItem} (objet de type \sphinxtitleref{roadmaps.Path})

\item {} 
\sphinxstyleliteralstrong{roads} \textendash{} Liste des routes si \sphinxtitleref{path} n’est pas défini(objet \sphinxtitleref{list} d’objet \sphinxtitleref{roadmap.Road})

\item {} 
\sphinxstyleliteralstrong{name} \textendash{} Nom de l’objet \sphinxtitleref{RoadItem}

\item {} 
\sphinxstyleliteralstrong{current\_time} \textendash{} Valeur initial du temps courant

\item {} 
\sphinxstyleliteralstrong{init\_time} \textendash{} Valeur du temps de début de scénario

\item {} 
\sphinxstyleliteralstrong{index} \textendash{} Valeur initial de la position dans l’itinéraire

\item {} 
\sphinxstyleliteralstrong{passable} \textendash{} Valeur initial de propriété de franchissement

\end{itemize}

\end{description}\end{quote}

\end{fulllineitems}

\index{\_\_le\_\_() (méthode road\_objects.road\_item.RoadItem)}

\begin{fulllineitems}
\phantomsection\label{\detokenize{road_objects/road_item:road_objects.road_item.RoadItem.__le__}}\pysiglinewithargsret{\sphinxbfcode{\_\_le\_\_}}{\emph{other}}{}
Test si self a une position avant un autre objet dans le parcours de \sphinxtitleref{self}.
\begin{quote}\begin{description}
\item[{Retourne}] \leavevmode
\begin{itemize}
\item {} 
True : si \sphinxtitleref{other} est sur une position à venir de l’itinéraire de \sphinxtitleref{self}

\item {} \begin{description}
\item[{False}] \leavevmode{[}si{]}\begin{itemize}
\item {} 
\sphinxtitleref{other} n’a pas de postion commune avec \sphinxtitleref{self}

\item {} 
\sphinxtitleref{other} n’a pas démarré

\item {} 
\sphinxtitleref{self} n’a pas démarré

\end{itemize}

\end{description}

\end{itemize}


\end{description}\end{quote}

\end{fulllineitems}

\index{add\_road() (méthode road\_objects.road\_item.RoadItem)}

\begin{fulllineitems}
\phantomsection\label{\detokenize{road_objects/road_item:road_objects.road_item.RoadItem.add_road}}\pysiglinewithargsret{\sphinxbfcode{add\_road}}{\emph{*road}}{}
Ajoute une ou plusieurs route(s) à l’itinairaire

\end{fulllineitems}

\index{delta\_time (attribut road\_objects.road\_item.RoadItem)}

\begin{fulllineitems}
\phantomsection\label{\detokenize{road_objects/road_item:road_objects.road_item.RoadItem.delta_time}}\pysigline{\sphinxbfcode{delta\_time}}
Durée entre deux mises à jour de \sphinxtitleref{current\_time}

\end{fulllineitems}

\index{distance() (méthode road\_objects.road\_item.RoadItem)}

\begin{fulllineitems}
\phantomsection\label{\detokenize{road_objects/road_item:road_objects.road_item.RoadItem.distance}}\pysiglinewithargsret{\sphinxbfcode{distance}}{\emph{position: roadmaps.position.Position}}{{ $\rightarrow$ float}}
Calcul la distance entre \textless{}self\textgreater{} et l’objet \textless{}item\textgreater{}

Pré-requis de construction : la distance entre chaque position est constante et vaut parameters.DISTANCE\_POSITION
distance ::=  (nb positions entre self et position) * DISTANCE\_POSITION

\(distance = \sum_{self}^{position}{nb\ positions}.(parameters.DISTANCE\_POSITION)\)
\begin{description}
\item[{\sphinxstyleemphasis{ancien mode de calcul :}}] \leavevmode
distance ::= SUM( DISTANCE(postion\_i, position\_i+1) ), avec position\_i in{[}self.position, item.position {[}
La fonction de calcul de distance entre 2 positions est défini par le fonction DISTANCE

\end{description}
\begin{quote}\begin{description}
\item[{Retourne}] \leavevmode
\begin{itemize}
\item {} 
distance(\sphinxtitleref{self}, \sphinxtitleref{other}) si \sphinxtitleref{other} a au moins une position commune avec \sphinxtitleref{self}

\item {} 
\sphinxtitleref{None} si pas de position commune

\end{itemize}


\end{description}\end{quote}

\end{fulllineitems}

\index{forward() (méthode road\_objects.road\_item.RoadItem)}

\begin{fulllineitems}
\phantomsection\label{\detokenize{road_objects/road_item:road_objects.road_item.RoadItem.forward}}\pysiglinewithargsret{\sphinxbfcode{forward}}{\emph{new\_time}}{}
Methode par pdéfaut pour faire avancer un objet dans le temps

\end{fulllineitems}

\index{get\_plot() (méthode road\_objects.road\_item.RoadItem)}

\begin{fulllineitems}
\phantomsection\label{\detokenize{road_objects/road_item:road_objects.road_item.RoadItem.get_plot}}\pysiglinewithargsret{\sphinxbfcode{get\_plot}}{\emph{new\_time=None}}{}
Retourne les éléménts graphique à afficher

\end{fulllineitems}

\index{get\_position() (méthode road\_objects.road\_item.RoadItem)}

\begin{fulllineitems}
\phantomsection\label{\detokenize{road_objects/road_item:road_objects.road_item.RoadItem.get_position}}\pysiglinewithargsret{\sphinxbfcode{get\_position}}{\emph{new\_time}}{}
Retourne la position (x,y) du Vehicule

\end{fulllineitems}

\index{init\_graphic() (méthode road\_objects.road\_item.RoadItem)}

\begin{fulllineitems}
\phantomsection\label{\detokenize{road_objects/road_item:road_objects.road_item.RoadItem.init_graphic}}\pysiglinewithargsret{\sphinxbfcode{init\_graphic}}{}{}
Initialise la représentation graphique

\end{fulllineitems}

\index{is\_ended (attribut road\_objects.road\_item.RoadItem)}

\begin{fulllineitems}
\phantomsection\label{\detokenize{road_objects/road_item:road_objects.road_item.RoadItem.is_ended}}\pysigline{\sphinxbfcode{is\_ended}}
Retourne \sphinxtitleref{True} si l’objet \sphinxtitleref{RoadItem} a fini son chemin sinon{}`False{}`

\end{fulllineitems}

\index{is\_started (attribut road\_objects.road\_item.RoadItem)}

\begin{fulllineitems}
\phantomsection\label{\detokenize{road_objects/road_item:road_objects.road_item.RoadItem.is_started}}\pysigline{\sphinxbfcode{is\_started}}
Retourne \sphinxtitleref{True} si l’objet \sphinxtitleref{RoadItem} a débuté son chemin sinon \sphinxtitleref{False}

\end{fulllineitems}

\index{length (attribut road\_objects.road\_item.RoadItem)}

\begin{fulllineitems}
\phantomsection\label{\detokenize{road_objects/road_item:road_objects.road_item.RoadItem.length}}\pysigline{\sphinxbfcode{length}}
Nombre de \sphinxtitleref{Position} sur l’itinéraire

\end{fulllineitems}

\index{next\_position (attribut road\_objects.road\_item.RoadItem)}

\begin{fulllineitems}
\phantomsection\label{\detokenize{road_objects/road_item:road_objects.road_item.RoadItem.next_position}}\pysigline{\sphinxbfcode{next\_position}}
Position estimée de la prochaine frame

\end{fulllineitems}

\index{passable (attribut road\_objects.road\_item.RoadItem)}

\begin{fulllineitems}
\phantomsection\label{\detokenize{road_objects/road_item:road_objects.road_item.RoadItem.passable}}\pysigline{\sphinxbfcode{passable}}
Retour le statut de franchissement

\end{fulllineitems}

\index{path (attribut road\_objects.road\_item.RoadItem)}

\begin{fulllineitems}
\phantomsection\label{\detokenize{road_objects/road_item:road_objects.road_item.RoadItem.path}}\pysigline{\sphinxbfcode{path}}
Itinéraire de l’objet \sphinxtitleref{RoadItem} (objet de type{}`Path{}`)

\end{fulllineitems}

\index{position (attribut road\_objects.road\_item.RoadItem)}

\begin{fulllineitems}
\phantomsection\label{\detokenize{road_objects/road_item:road_objects.road_item.RoadItem.position}}\pysigline{\sphinxbfcode{position}}
Position courante de l’objet \sphinxtitleref{RoadItem}
Retourne \sphinxtitleref{None} s’il n’est pas sur le chemin (pas actif)

\end{fulllineitems}

\index{remain\_path() (méthode road\_objects.road\_item.RoadItem)}

\begin{fulllineitems}
\phantomsection\label{\detokenize{road_objects/road_item:road_objects.road_item.RoadItem.remain_path}}\pysiglinewithargsret{\sphinxbfcode{remain\_path}}{\emph{end\_index=None}}{}
Retourne la portion renatnt à parcourrir pour l’objet \sphinxtitleref{RoadItem}
\begin{quote}\begin{description}
\item[{Paramètres}] \leavevmode
\sphinxstyleliteralstrong{end\_index} \textendash{} Index correspondant à la borne max

\item[{Retourne}] \leavevmode
\begin{itemize}
\item {} 
Objet \sphinxtitleref{list} contenant les objets \sphinxtitleref{Position} à venir s’il en reste.

\item {} 
Objet \sphinxtitleref{list} vide s’il n’y a plus rien à parcourrir

\end{itemize}


\end{description}\end{quote}

\end{fulllineitems}

\index{set\_impassable() (méthode road\_objects.road\_item.RoadItem)}

\begin{fulllineitems}
\phantomsection\label{\detokenize{road_objects/road_item:road_objects.road_item.RoadItem.set_impassable}}\pysiglinewithargsret{\sphinxbfcode{set\_impassable}}{}{}
Rend RoadItem infranchissable

\end{fulllineitems}

\index{set\_passable() (méthode road\_objects.road\_item.RoadItem)}

\begin{fulllineitems}
\phantomsection\label{\detokenize{road_objects/road_item:road_objects.road_item.RoadItem.set_passable}}\pysiglinewithargsret{\sphinxbfcode{set\_passable}}{\emph{mode=True}}{}
Change le statut de franchissement:
- True : rend RoadItem franchissable (par défaut)
- False : rend RoadItem infranchissable

\end{fulllineitems}

\index{start() (méthode road\_objects.road\_item.RoadItem)}

\begin{fulllineitems}
\phantomsection\label{\detokenize{road_objects/road_item:road_objects.road_item.RoadItem.start}}\pysiglinewithargsret{\sphinxbfcode{start}}{\emph{init\_time}}{}
Définit le moment du départ : permet un décalage dans la lecture des positions

\end{fulllineitems}


\end{fulllineitems}



\chapter{Indices and tables}
\label{\detokenize{index:indices-and-tables}}\begin{itemize}
\item {} 
\DUrole{xref,std,std-ref}{genindex}

\item {} 
\DUrole{xref,std,std-ref}{modindex}

\item {} 
\DUrole{xref,std,std-ref}{search}

\end{itemize}


\section{Package \sphinxtitleref{road\_objects}}
\label{\detokenize{index:package-road-objects}}\label{\detokenize{index:module-road_objects.vehicule}}\index{road\_objects.vehicule (module)}
Ce module contient la définition de la classe Vehicule
\begin{itemize}
\item {} 
Vehicule.index : cast la valeur en “int”

\end{itemize}
\index{Vehicule (classe dans road\_objects.vehicule)}

\begin{fulllineitems}
\phantomsection\label{\detokenize{index:road_objects.vehicule.Vehicule}}\pysiglinewithargsret{\sphinxbfcode{class }\sphinxcode{road\_objects.vehicule.}\sphinxbfcode{Vehicule}}{\emph{axe}, \emph{path=None}, \emph{name=None}}{}
Regroupe les routes  de son itinairaire
et définit la position du véhicule dans le temps
\index{List\_vehicules() (méthode de la classe road\_objects.vehicule.Vehicule)}

\begin{fulllineitems}
\phantomsection\label{\detokenize{index:road_objects.vehicule.Vehicule.List_vehicules}}\pysiglinewithargsret{\sphinxbfcode{classmethod }\sphinxbfcode{List\_vehicules}}{}{}
Retourne la liste des véhicules

\end{fulllineitems}

\index{get\_plot() (méthode road\_objects.vehicule.Vehicule)}

\begin{fulllineitems}
\phantomsection\label{\detokenize{index:road_objects.vehicule.Vehicule.get_plot}}\pysiglinewithargsret{\sphinxbfcode{get\_plot}}{\emph{new\_time=None}}{}
Retourne les éléménts graphique à afficher

\end{fulllineitems}

\index{update\_speed() (méthode road\_objects.vehicule.Vehicule)}

\begin{fulllineitems}
\phantomsection\label{\detokenize{index:road_objects.vehicule.Vehicule.update_speed}}\pysiglinewithargsret{\sphinxbfcode{update\_speed}}{\emph{distance: float = None}}{{ $\rightarrow$ None}}
Calcul et met à jour la vitesse en fonction de la distance de l’objet routier suivant
:param distance: distance entre \textless{}self\textgreater{} et l’objet suivant

\end{fulllineitems}


\end{fulllineitems}



\section{Package \sphinxtitleref{roadmaps}}
\label{\detokenize{index:package-roadmaps}}\label{\detokenize{index:module-roadmaps.map}}\index{roadmaps.map (module)}
Project name : TIPE\_Yoann
Module name : map.py

Classes list in this module:
- Map : Classe de base pour les cartes
————————————————————————————————————————
Author : Eric Ollivier
Create date : 20/02/2021
————————————————————————————————————————
Versionning :
0.1 : Initial version

\phantomsection\label{\detokenize{index:module-roadmaps.path}}\index{roadmaps.path (module)}
Project name : TIPE\_Yoann
Module name : path.py

Classes list in this module:
- Path
————————————————————————————————————————
Author : Eric Ollivier
Create date : 14/02/2021
————————————————————————————————————————
Versionning :
0.1 : Initial version
0.2 :
- Ajout de la méthode “append\_position” permettant d’ajouter une position à un chemin

\phantomsection\label{\detokenize{index:module-roadmaps.road}}\index{roadmaps.road (module)}
Ce module contient les routes permettant au véhicule de se déplacer
- Class road
- variable roadmap
\index{Road (classe dans roadmaps.road)}

\begin{fulllineitems}
\phantomsection\label{\detokenize{index:roadmaps.road.Road}}\pysiglinewithargsret{\sphinxbfcode{class }\sphinxcode{roadmaps.road.}\sphinxbfcode{Road}}{\emph{x\_interval}, \emph{y\_interval}, \emph{path\_functions}, \emph{step: int = None}}{}
Définit la fonction de calcul de la trajectoire entre deux positions.
\begin{quote}\begin{description}
\item[{Paramètres}] \leavevmode\begin{itemize}
\item {} 
\sphinxstyleliteralstrong{x\_interval} \textendash{} Intervalle de calcul (x\_start, x\_end)

\item {} 
\sphinxstyleliteralstrong{y\_interval} \textendash{} Intervalle de calcul (y\_start, y\_end)

\item {} 
\sphinxstyleliteralstrong{path\_functions} \textendash{} Fonctions paramétriques (x\_func, y\_func)
de calcul de la trajectoire entre les deux positions

\end{itemize}

\end{description}\end{quote}
\index{init\_step() (méthode roadmaps.road.Road)}

\begin{fulllineitems}
\phantomsection\label{\detokenize{index:roadmaps.road.Road.init_step}}\pysiglinewithargsret{\sphinxbfcode{init\_step}}{}{}
Calcul le nombre de pas pour avoir une distance de DISTANCE\_POSTION

\end{fulllineitems}

\index{setup\_path() (méthode roadmaps.road.Road)}

\begin{fulllineitems}
\phantomsection\label{\detokenize{index:roadmaps.road.Road.setup_path}}\pysiglinewithargsret{\sphinxbfcode{setup\_path}}{}{}
Calcule les coordonnées de la route

\end{fulllineitems}


\end{fulllineitems}

\phantomsection\label{\detokenize{index:module-roadmaps.position}}\index{roadmaps.position (module)}
Project name : TIPE\_Yoann
Module name : position.py

Classes list in this module:
- Position
————————————————————————————————————————
Author : Eric Ollivier
Create date : 14/02/2021
————————————————————————————————————————
Versionning :
0.1 : Initial version
\index{Position (classe dans roadmaps.position)}

\begin{fulllineitems}
\phantomsection\label{\detokenize{index:roadmaps.position.Position}}\pysiglinewithargsret{\sphinxbfcode{class }\sphinxcode{roadmaps.position.}\sphinxbfcode{Position}}{\emph{x}, \emph{y}}{}
Classe définissant une position pour un objet RoadItem

\end{fulllineitems}

\phantomsection\label{\detokenize{index:module-roadmaps.traffic_circle}}\index{roadmaps.traffic\_circle (module)}
Carte d’un rond-point


\renewcommand{\indexname}{Index des modules Python}
\begin{sphinxtheindex}
\def\bigletter#1{{\Large\sffamily#1}\nopagebreak\vspace{1mm}}
\bigletter{r}
\item {\sphinxstyleindexentry{road\_objects.graphical\_item}}\sphinxstyleindexpageref{road_objects/graphical_item:\detokenize{module-road_objects.graphical_item}}
\item {\sphinxstyleindexentry{road\_objects.road\_item}}\sphinxstyleindexpageref{road_objects/road_item:\detokenize{module-road_objects.road_item}}
\item {\sphinxstyleindexentry{road\_objects.vehicule}}\sphinxstyleindexpageref{index:\detokenize{module-road_objects.vehicule}}
\item {\sphinxstyleindexentry{roadmaps.map}}\sphinxstyleindexpageref{index:\detokenize{module-roadmaps.map}}
\item {\sphinxstyleindexentry{roadmaps.path}}\sphinxstyleindexpageref{index:\detokenize{module-roadmaps.path}}
\item {\sphinxstyleindexentry{roadmaps.position}}\sphinxstyleindexpageref{index:\detokenize{module-roadmaps.position}}
\item {\sphinxstyleindexentry{roadmaps.road}}\sphinxstyleindexpageref{index:\detokenize{module-roadmaps.road}}
\item {\sphinxstyleindexentry{roadmaps.traffic\_circle}}\sphinxstyleindexpageref{index:\detokenize{module-roadmaps.traffic_circle}}
\end{sphinxtheindex}

\renewcommand{\indexname}{Index}
\printindex
\end{document}